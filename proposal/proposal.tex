\documentclass[a4,12pt]{article}
\usepackage{amsmath}
\usepackage{amssymb}
\usepackage[sorting=none]{biblatex}
\usepackage{hyperref}
\usepackage{graphicx}
% limit page margin
\usepackage[margin=1.2in]{geometry}

\addbibresource{reference.bib}
\title{Proposal for PhD Research: Quantifying the Uncertainty in both On and Off-target Activity of Prime Editing}
\author{}
\date{}

\begin{document}
\maketitle

My previous work on the master's thesis focused on the on target prediction of prime editing efficiency, however, the off-target activity of prime editing is still a major concern. 

With the recent development of a number of off-target site detection protocols supporting prime editors, it is now possible to quantify their off target activity using big data and deep learning techniques\parencite{liangGenomewideProfilingPrime2023,
zhuTrackingseqRevealsHeterogeneity2024}.

Although majority of the existing solution (including my master's thesis) was aiming at producing a point estimation as the predicted editing efficiency\parencite{mathisMachineLearningPrediction2024,yuPredictionEfficienciesDiverse2023,koeppelPredictionPrimeEditing2023}, prime editing itself must be viewed as a stochastic process due to the physical complexity of the molecular interactions involved. Thus, instead of producing a point estimation, it could be more informative to model the posterior distribution of the activity of prime editing given the target loci and pegRNA sequence.

A reasonable starting point would be to investigate 

Some preliminary study has been done on the  

MC dropout/Deep ensemble

\printbibliography

\end{document}